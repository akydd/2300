\section{The traditional Adventist interpretation}
The traditional Adventist interpretations of the four visions are provided here without explanation.
They are taken from todo.

\subsection{Daniel's first vision}

\begin{verse}
    \poemlines{1}
     In the first year of King Belshazzar of Babylon, Daniel had a dream filled with visions while he was lying on his bed. Then he wrote down the dream in summary fashion.\\
     Daniel explained: ``I was watching in my vision during the night as the four winds of the sky were stirring up the great sea.\\
     Then four large beasts came up from the sea; they were different from one another.\\
     ``The first one was like a lion with eagles’ wings. As I watched, its wings were pulled off, and it was lifted up from the ground. It was made to stand on two feet like a human being, and a human mind was given to it. \\
     ``Then a second beast appeared, like a bear. It was raised up on one side, and there were three ribs in its mouth between its teeth. It was told, `Get up and devour much flesh!’ \\
     ``After these things, as I was watching, another beast like a leopard appeared, with four bird-like wings on its back. This beast had four heads, and ruling authority was given to it. \\
     ``After these things, as I was watching in the night visions a fourth beast appeared—one dreadful, terrible, and very strong. It had two large rows of iron teeth. It devoured and crushed, and anything that was left it trampled with its feet. It was different from all the beasts that came before it, and it had 10 horns. \\
     ``As I was contemplating the horns, another horn-—a small one-—came up between them, and three of the former horns were torn out by the roots to make room for it. This horn had eyes resembling human eyes and a mouth speaking arrogant things. \\
     ``While I was watching, thrones were set up, and the Ancient of Days took his seat. His attire was white like snow; the hair of his head was like lamb’s wool. His throne was ablaze with fire, and its wheels were all aflame. \\
     A river of fire was streaming forth and proceeding from his presence. Many thousands were ministering to him; many tens of thousands stood ready to serve him. The court convened, and the books were opened. \\
     ``Then I kept on watching because of the arrogant words of the horn that was speaking. I was watching until the beast was killed and its body destroyed and thrown into the flaming fire. \\
     As for the rest of the beasts, their ruling authority had already been removed, though they were permitted to go on living for a time and a season. \\
\end{verse}

\begin{center}
    \begin{tabularx}{\textwidth}{@{}XX@{}}
        \toprule
        \textbf{Symbol} & \textbf{Interpretation} \\
        \midrule
        First beast, the lion with eagles wings & Babylon \\
        Second beast, the lopsided bear & Media-Persia \\
        Third beast, the four-headed and winged leopard & Greece \\
        Fourth beast with 10 horns & pagan Rome \\
        Little horn that arises after the 10 horns & papal Rome \\
        ``time, times, and half a time'' during which the ``saints of the Most High'' are ``given unto his hand'' & period of 1260
        years of papal supremacy, 538\AD--1798\AD. \\
        \bottomrule
    \end{tabularx}
\end{center}

\subsection{Daniel's second vision}

\begin{center}
    \begin{tabularx}{\textwidth}{@{}XX@{}}
        \toprule
        \textbf{Symbol} & \textbf{Interpretation} \\
        \midrule
        Ram &  Media-Persia \\
        Goat & Greece \\
        Goat's single horn & Alexander the Great \\
        Four horns arising after the great horn is broken & four of Alexander's generals who ruled over the divided kingdom after Alexander's death in 323\BC \\
        Little horn that arises after the four & Rome, pagan initially and then papal \\
        Removing of the daily sacrifice and casting down of the sanctuary &  symbolic of papal Rome's instition of intercession by priests as a counterfiet worship system to Christ's intercession ministry in the heavenly sanctuary \\
        2300 days until the sanctuary is cleansed & a 2300 year time period that ends with a heavenly counterpart to the earthly temple's service at the day of atonement.  There is not enough revealed in this chapter to determine a starting or ending for this time period. \\
        \bottomrule
    \end{tabularx}
\end{center}

\subsection{Daniel's third vision}

The decree to rebuild Jerusalem is taken at 457 BC.

The seventy ``weeks'' of Daniel 9 are thus 490 years, and are taken to start at the decree to rebuild Jerusalem.  This is split up into three periods.
The first seven weeks and following 62 weeks mentioned in Daniel 9:25 are combined into a 483 period that ends with the baptism of Jesus in 27 AD.

These 490 years are taken to be the first of the 2300 years in the previous vision.  Thus the 2300 years ends in 1844.

The 70th week is a period of 7 years following the baptism of Jesus.  He was crucified 3.5 years into his public ministry, the middle of the week.
His death on the cross is the ``end to sacrifice and offering''.  The end of the 70th week is the stoning of Stephen.

\subsection{Analysis}
Assuming that the above interpretation is without error, are there any discrepancies that arise with elements of the visions which were not
directly used in the interpretation?
