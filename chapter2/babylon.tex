\section{Neo-Babylonian Empire}
Babylonia was founded as an independent state in 1894 BC\@. It's power grew into the First
Babylonian Empire under Hammurabi, who ruled from 1792--1750 BC\@. It was incorporated into the Neo-Assyrian Empire in 798 BC.

General Nabopolassar took advantage of a civil war within the Neo-Assyrian Empire and revolted, formally becoming King
of Babylon on 22 or 23 of November, 626 BC\@. He eventually allied with king Cyraxes of the Medes in 612 BC, and other groups,
to finally end the Neo-Assyrian Empire in 609 BC after the seige of Harram, the last capital of the Neo-Assyrian Empire.


\begin{center}
    \begin{tabularx}{\textwidth}{@{}rX@{}}
        \toprule
        \textbf{Date} & \textbf{Event} \\
        \midrule
        22 or 23 November 626 BC & Nabopolassar becomes king \\
        605 BC & Nebuchadnezzar II becomes king \\
        562 BC & Amel-Marduk becomes king \\
        560 BC & Neriglissar becomes king \\
        556 BC & Labashi-Marduk becomes king and is asassinated; Nabonidus becomes king \\
        553 BC & Nabonidus leaves for Teima; Belshazzar rules as co-regent \\
        543 BC & Nabonidus returns \\
        12 October 539 BC & Cyrus the Great enters Babylon \\
        \bottomrule
    \end{tabularx}
\end{center}
