\documentclass[]{article}

\begin{document}
\begin{abstract}
Here is the abstract
\end{abstract}

\title{2300 days}

\maketitle

\section{Introduction}

\section{Historical background}

\subsection{Neo-Babylonian Empire}
\begin{center}
\begin{tabular}{ll}
\hline
Date & Event \\
\hline\hline
626 BC & Nabopolassar becomes king \\
605 BC & Nebuchadnezzar II becomes king \\
562 BC & Amel-Marduk becomes king \\
560 BC & Neriglissar becomes king \\
556 BC & Labashi-Marduk becomes king and is asassinated; Nabonidus becomes king \\
553 BC & Nabonidus leaves for Teima; Belshazzar rules as co-regent \\
543 BC & Nabonidus returns \\
12 October 539 BC & Cyrus the Great enters Babylon \\
\end{tabular}
\end{center}


\subsection{Persian Empire}

\begin{center}
\begin{tabular}{ll}
\hline
Date & Event \\
\hline\hline
559 BC & Cyrus becomes king of Persia \\
549 BC & Cyrus becomes king of Media \\
539 BC & Cyrus becomes king of Babylon \\
530 BC & Cambyses becomes king \\
522 BC & Bardiya becomes king; Darius I becomes king \\
486 BC & Xerxes I becomes king \\
465 BC & Artaxerxes I becomes king \\
424 BC & Xeres II becomes king \\
424 BC & Sogidanius becomes king \\
423 BC & Darius II becomes king \\
404 BC & Artaxerxes II becomes king \\
358 BC & Artaxerxes III becomes king \\
338 BC & Arses becomes king \\
336 BC & Darius III becomes king \\
\end{tabular}
\end{center}


\subsection{Greek kingdom of Macedon}


\subsection{Return to Zion}
The Biblical account of the Jews return to to the land if Israel from the Babylonian exile is contained
in the books of Ezra and Nehemiah. Notable events within this account include the decree by emporor Cyrus the Great, also knows as Cyrus's edict, and the
reconstruction works of the city of Jerusalem and of the second temple.

Combining the historical records with the Biblical account, the events and their approximate dates are as follows:

\begin{center}
\begin{tabular}{lll}
\hline
    Date & Event & Reference \\
\hline\hline
    539 BC & 1st year of Cyrus, King of Persia: edict to build the temple & Ezra 1:1--4 \\
    1st day of the 7th month & altar rebuilt & Ezra 3:6 \\
    2nd month of the 2nd year after arrival at the house of God in Jerusalem & work on the temple begins & Ezra 3:8 \\
    520 BC 2nd year of Darius king of Persia & building resumes by Zerubbabel and Joshua & Ezra 5:2 \\
     & Opposition, defence stating the previous edict of Cyrus in the 1st year of his reign & \\
    & Darius replies citing the earlier edict & \\
    515 BC 3rd of Adar, in the 6th year of King Darius reign & Temple completed & Ezra 6:15 \\
    14th day of the 1st month & Passover & Ezra 6:19 \\
    486 BC Beginning of reign of Xerxes & accusation & Ezra 4:6 \\
    465 BC In the days of Artaxerxes king of Persia & letter to Artaxerxes re rebuilding city walls and foundations & Ezra 4:6--23 \\
     & Artaxerxes replies commanding work to stop & \\
     & Rebuilding stops & \\
    459 BC 5th month of the 7th year of King Artexerxes & Ezra arrives in Jerusalem & \\
    445 BC Month of Nissan in the 20th year of King Artaxerxes & Nehemiah requests to return to Judah to rebuild Jerusalem & Nehemiah 2:1--5 \\
     & Sometime after this construction on the wall begins & \\
    Elui 5 & wall completed after 52 days? & Nehemiah 6:15 \\
    434 BC 32nd year of King Artaxerxes & Nehemiah returns to Babylon & Nehemiah 13:6 \\
\end{tabular}
\end{center}


\subsection{Daniel's visions}
Daniel's prophetic visions are all contained within the book of Daniel.  According to tbe Biblical account, they span a xx year period, starting during the reign of King Belshazzar
and ending during the reign of King Cyrus.

In chronological order, according to the Biblical account:
\begin{enumerate}
    \item Daniel's first vision (Daniel 7): First year of King Belshazzar of Babylon  4 beasts, 3.5 years for the horn of the 4th beast
    \item Daniel's second vision (Daniel 8): Third year of King Belshazzar of Babylon Ram and the goat, daily sacrifice taken away for 2300 evenings--mornings
    \item Daniel's third vision (Daniel 9): First year of Darius son of Xerxes (or Ahasuerus)  70 7s
    \item Daniel's fourth vision (Daniel 10--12): Third year of Cyrus King of Persia, 24th day of the 1st month  Son of Man, King of the North
\end{enumerate}

Combining this with the historical accounts from the previou sections is problematic as there is no extra-Biblical historical reference to King Darius, whom,
according to the text of the third vision, is the son of Ahasuerus and a Mede by descent.

Leaving the dating of the third vision aside for the moment, we arrive at the following dates for the visions:
\begin{enumerate}
    \item Daniel's first vision: 553 BC
    \item Daniel's second vision: 551 BC
    \item Deniel's third vision: unknown
    \item Daniel's fourth vision: 537 BC
\end{enumerate}

\section{The traditional Adventist interpretation}
The traditional Adventist interpretations of the four visions are provided here without explanation.
They are taken from todo.

\subsection{Daniel's first vision}
The four beasts from the first vision are interpreted thusly.

The first beast, the lion with eagles wings, is Babylon.

The second beast, the lopsided bear, is Media-Persia.

The third beast, the four-headed and winged leopard, is Greece.

The fourth beast, with 10 horns, is ``pagan Rome''.

The little horn that arises after the 10 is papal Rome (or the papacy?).

The ``time, times, and half a time'' during which the ``saints of the Most High'' are ``given unto his hand'' is a period of 1260
years of papal supremacy starting in 538 AD and ending in 1798 AD.

\subsection{Daniel's second vision}
The ram is Media-Persia.

The goat is Greece, with the single horn representing Alexander the Great.  The four horns arising after the great horn is broken are
four of Alexander's generals who ruled over the divided kingdom after Alexander's death in 323 BC.

The little horn that arises after the four (Daniel 8:9--12) is Rome, pagan initially and then papal.

The removing of the daily sacrifice and casting down of the sanctuary are symbolic of papal Rome's instition of intercession by priests
as a counterfiet worship system to Christ's intercession ministry in the heavenly sanctuary.

The 2300 days until the sanctuary is cleansed is a 2300 year time period that ends with a heavenly counterpart to the earthly temple's
service at the day of atonement.  There is not enough revealed in this chapter to determine a starting or ending for this time period.

\subsection{Daniel's third vision}

The decree to rebuild Jerusalem is taken at 457 BC.

The seventy ``weeks'' of Daniel 9 are thus 490 years, and are taken to start at the decree to rebuild Jerusalem.  This is split up into three periods.
The first seven weeks and following 62 weeks mentioned in Daniel 9:25 are combined into a 483 period that ends with the baptism of Jesus in 27 AD.

These 490 years are taken to be the first of the 2300 years in the previous vision.  Thus the 2300 years ends in 1844.

The 70th week is a period of 7 years following the baptism of Jesus.  He was crucified 3.5 years into his public ministry, the middle of the week.
His death on the cross is the ``end to sacrifice and offering''.  The end of the 70th week is the stoning of Stephen.

\section{Analysis}
Assuming that the above interpretation is without error, are there any discrepancies that arise with elements of the visions which were not
directly used in the interpretation?



\end{document}
