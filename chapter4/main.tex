\chapter{The traditional Adventist interpretation}
The traditional Adventist interpretations of the four visions are provided here without explanation.
They are taken from todo.

\section{Daniel's first vision}

\begin{center}
    \begin{tabularx}{\textwidth}{@{}XX@{}}
        \toprule
        \textbf{Symbol} & \textbf{Interpretation} \\
        \midrule
        First beast, the lion with eagles wings & Babylon \\
        Second beast, the lopsided bear & Media-Persia \\
        Third beast, the four-headed and winged leopard & Greece \\
        Fourth beast with 10 horns & pagan Rome \\
        Little horn that arises after the 10 horns & papal Rome \\
        ``time, times, and half a time'' during which the ``saints of the Most High'' are ``given unto his hand'' & period of 1260
        years of papal supremacy, 538\AD--1798\AD. \\
        \bottomrule
    \end{tabularx}
\end{center}

\section{Daniel's second vision}

\begin{center}
    \begin{tabularx}{\textwidth}{@{}XX@{}}
        \toprule
        \textbf{Symbol} & \textbf{Interpretation} \\
        \midrule
        Ram &  Media-Persia \\
        Goat & Greece \\
        Goat's single horn & Alexander the Great \\
        Four horns arising after the great horn is broken & four of Alexander's generals who ruled over the divided kingdom after Alexander's death in 323\BC \\
        Little horn that arises after the four & Rome, pagan initially and then papal \\
        Removing of the daily sacrifice and casting down of the sanctuary &  symbolic of papal Rome's instition of intercession by priests as a counterfiet worship system to Christ's intercession ministry in the heavenly sanctuary \\
        2300 days until the sanctuary is cleansed & a 2300 year time period that ends with a heavenly counterpart to the earthly temple's service at the day of atonement.  There is not enough revealed in this chapter to determine a starting or ending for this time period. \\
        \bottomrule
    \end{tabularx}
\end{center}

\section{Daniel's third vision}

The decree to rebuild Jerusalem is taken at 457 BC.

The seventy ``weeks'' of Daniel 9 are thus 490 years, and are taken to start at the decree to rebuild Jerusalem.  This is split up into three periods.
The first seven weeks and following 62 weeks mentioned in Daniel 9:25 are combined into a 483 period that ends with the baptism of Jesus in 27 AD.

These 490 years are taken to be the first of the 2300 years in the previous vision.  Thus the 2300 years ends in 1844.

The 70th week is a period of 7 years following the baptism of Jesus.  He was crucified 3.5 years into his public ministry, the middle of the week.
His death on the cross is the ``end to sacrifice and offering''.  The end of the 70th week is the stoning of Stephen.

\section{Analysis}
Assuming that the above interpretation is without error, are there any discrepancies that arise with elements of the visions which were not
directly used in the interpretation?
